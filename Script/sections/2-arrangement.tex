\section{第2サイクルに入る前の整理} \label{sec:arrangement}

ここは議事録のようなもの.

\subsection{第1サイクルの事後分析について}

これまで作成した成果物は次のとおり.

\begin{itemize}
  \item 概念設計
  \item 背景と目的
  \item 用語定義
  \item 要件定義
  \item データ要件(ログ)
  \item 技術要件
  \item アプリ形態(クロスプロットフォーム)
  \item アーキテクチャ図
  \item 画面設計
  \item 機能のシーケンス図
\end{itemize}

計画〜基本設計のフェーズに当てはめると、次のようになる.なお内容が不十分だと思われるものには「?」をつけた.

\begin{table}[htbp]
  \centering
  \begin{tabular}{|l|l|}
  \hline
    計画 &
    \begin{tabular}{l}
    概念設計 \\
    \end{tabular} \\
  \hline
    要求定義 &
    \begin{tabular}{l}
      背景と目的(?) \\
      \quad •ゼミで指摘されたため内容不十分の可能性あり \\
      \quad •レポートで表示するグラフが適切かどうか \\
      用語定義(?) \\
      \quad •初期に作成したので内容不十分の可能性あり \\
      要件定義 \\
      データ要件(ログ) \\
      技術要件 \\
      アプリ形態(クロスプロットフォーム) \\
      \end{tabular} \\
  \hline
    基本設計 &
    \begin{tabular}{l}
      アーキテクチャ図 \\
      画面設計(?) \\
      機能のシーケンス図 \\
      \end{tabular} \\
  \hline
  \end{tabular}
\end{table}

事後分析は以上とする.

% 構成管理計画

\subsection{第1サイクルの戦略について}

\subsubsection{概念設計}

現時点で考えられる機能は次のとおり. なお要求と並べているのは、要求を満たす機能であるかどうかを確認するため.

\begin{table}[htbp]
  \centering
  \begin{tabular}{|l|l|}
  \hline
    要求 & 機能 \\
  \hline

  \hline
    ログを取りたい &
    \begin{tabular}{l}
      ログを計測する画面 \\
      \quad •開始時刻を取得 \\
      \quad •終了時刻を取得 \\
      \quad •ログ情報を取得し、保存 \\
      ログを編集する機能 \\
      \quad •更新されたログ情報を保存 \\
      ログを削除する機能 \\
      \quad •指定されたログを削除
      ユーザーごとにログを識別する機能 \\
      \quad •サインアップ \\
      \quad •サインイン \\
      \quad •ユーザー情報の編集 \\
    \end{tabular} \\
  \hline
    ログを閲覧したい &
    \begin{tabular}{l}
      デイリーレポートを表示する画面 \\
      \quad •日付を指定し、その日のログを表示 \\
      \quad •その日の統計情報を表示 \\
      \quad •その日のログについてのグラフを表示 \\
      週間レポートを表示する画面 \\
      \quad •週を指定し、その週のログを表示 \\
      \quad •その週の統計情報を表示 \\
      \quad •その週のログについてのグラフを表示 \\
      月間レポートを表示する画面 \\
      \quad •月を指定し、その月のログを表示 \\
      \quad •その月の統計情報を表示 \\
      \quad •その月のログについてのグラフを表示 \\
      エクスポート機能 \\
      \quad •各レポートからデータをエクスポート \\
    \end{tabular} \\
  \hline
    ログをとる習慣をつけたい &
    \begin{tabular}{l}
      リマインド機能 \\
      \quad •ログ計測の開始リマインドを送信 \\
      \quad •ログ計測の終了リマインドを送信 \\
      \quad •リマインドの設定 \\
    \end{tabular} \\
  \hline
  \end{tabular}
\end{table}

表から、画面と機能を整理すると次のようになる.

\begin{table}[H]
  \centering
  \caption{機能}
  \begin{tabular}{|l|l|}
  \hline
    機能 & LOC \\
  \hline

  \hline
    開始時刻を取得 & 30 + 0 = 30 \\
  \hline
    終了時刻を取得 & 30 + 0 = 30 \\
  \hline
    ログ情報を取得し、保存 & 120 + 20 = 140 \\
  \hline
    更新されたログ情報を保存 & 60 + 50 = 110 \\
  \hline
    指定されたログを削除 & 60 + 50 = 110 \\
  \hline

  \hline
    ある期間を指定し、その期間のログを表示 & 20 + 50 = 70 \\
  \hline
    ある期間の統計情報を表示 & 40 + 20 = 60 \\
  \hline
    ある期間のログについてのグラフを表示 & 100 + 0 = 100 \\
  \hline
    各レポートからデータをエクスポート & 40 + 0 = 40 \\
  \hline

  \hline
    ログ計測の開始リマインドを送信 & 20 + 0 = 20 \\
  \hline
    ログ計測の終了リマインドを送信 & 20 + 0 = 20 \\
  \hline
    リマインドの設定 & 250 + 50 = 300 \\
  \hline

  \hline
    サインアップ &  90 + 30 = 120 \\
  \hline
    サインイン & 90 + 30 = 120 \\
  \hline
    ユーザー情報の編集 & 100 + 40 = 140 \\
  \hline
  \end{tabular}
\end{table}

\begin{table}[H]
  \centering
  \caption{想定画面}
  \begin{tabular}{|l|}
  \hline
    サインアップ画面  \\
    サインイン画面 \\
    計測開始画面 \\
    計測中画面 \\
    計測終了画面 \\
    設定画面 \\
    デイリーレポート画面 \\
    週間レポート画面 \\
    月間レポート画面 \\
  \hline
  \end{tabular}
\end{table}

バックエンド
フロントエンド機能
UI


\subsubsection{時間見積もり}

想定画面について、1画面あたり2時間とすると、デザインにかかる時間は $9 * 12 = 108$ 時間

また、チーム 2 LOC/hour で開発できると仮定すると、上記の機能は次のような時間がかかる.

\begin{table}[H]
  \centering
  \caption{機能}
  \begin{tabular}{|l|l|}
  \hline
    機能 & 時間見積もり(h) \\
  \hline

  \hline
    開始時刻を取得 & 10 \\
  \hline
    終了時刻を取得 & 10 \\
  \hline
    ログ情報を取得し、保存 & 46.6 \\
  \hline
    更新されたログ情報を保存 & 36.6 \\
  \hline
    指定されたログを削除 & 36.6 \\
  \hline

  \hline
    ある期間を指定し、その期間のログを表示 & 13.3 \\
  \hline
    ある期間の統計情報を表示 & 20 \\
  \hline
    ある期間のログについてのグラフを表示 & 33.3 \\
  \hline
    各レポートからデータをエクスポート & 13.3 \\
  \hline

  \hline
    ログ計測の開始リマインドを送信 & 6.6 \\
  \hline
    ログ計測の終了リマインドを送信 & 6.6 \\
  \hline
    リマインドの設定 & 100 \\
  \hline

  \hline
    サインアップ & 40 \\
  \hline
    サインイン & 40 \\
  \hline
    ユーザー情報の編集 & 46.6 \\
  \hline
  \end{tabular}
\end{table}

よって、プロジェクトの時間見積もりは $108 + 459.5 = 567.5$ 時間となる.

1人あたりの1週間の平均作業時間が14時間のため、プロジェクトの期間は $567.5 / (14 * 2) = 20.3$ 週となる. \\

すなわち、約4.8ヶ月弱ほどかかると見積もられる.


\subsubsection{サイクル分割}

当初設定したサイクル分割は次のとおりであった.

\begin{table}[H]
  \centering
  \begin{tabular}{|l|l|}
  \hline
    第1サイクル &
    \begin{tabular}{l}
      ログを計測する機能 \\
      ログを編集・削除する機能 \\
      デイリーレポートを表示する機能 \\
      リマインド機能 \\
      サインアップ・サインイン
    \end{tabular} \\
  \hline
    第2サイクル &
    \begin{tabular}{l}
      週間レポートを表示する機能 \\
      月間レポートを表示する機能 \\
      エクスポート機能
    \end{tabular} \\
  \hline
    第3サイクル &
    \begin{tabular}{l}
      ユーザー情報の編集機能 \\
      カテゴリのレコメンド機能
    \end{tabular} \\
  \hline
    第4サイクル &
    \begin{tabular}{l}
      ユーザー情報の公開機能 \\
    \end{tabular} \\
  \hline
  \end{tabular}
\end{table}

時間見積もりを参考に、サイクル分割を見直す.

\begin{table}[H]
  \centering
  \begin{tabular}{|l|l|l|}
  \hline
    サイクル & 機能 & かかる時間(h) \\
  \hline

  \hline
    第1サイクル &
    \begin{tabular}{l}
      開始時刻を取得 \\
      終了時刻を取得 \\
      ログ情報を取得し、保存 \\
    \end{tabular} &
    66.6 \\
  \hline
    第2サイクル &
    \begin{tabular}{l}
      ある期間を指定し、その期間のログを表示 \\
      ある期間の統計情報を表示 \\
      ある期間のログについてのグラフを表示 \\
    \end{tabular} &
    66.6 \\
  \hline
    第3サイクル &
    \begin{tabular}{l}
      更新されたログ情報を保存 \\
      指定されたログを削除 \\
    \end{tabular} &
    73.2 \\
  \hline
    第4サイクル &
    \begin{tabular}{l}
      リマインドの設定
    \end{tabular} &
    100 \\
  \hline
    第5サイクル &
    \begin{tabular}{l}
      ログ計測の開始リマインドを送信 \\
      ログ計測の終了リマインドを送信 \\
    \end{tabular} &
    13.2 \\
  \hline
    第6サイクル &
    \begin{tabular}{l}
      サインアップ \\
      サインイン \\
    \end{tabular} &
    80 \\
  \hline
    第7サイクル &
    \begin{tabular}{l}
      ユーザー情報の編集 \\
      各レポートからデータをエクスポート \\
    \end{tabular} &
    59.9 \\
  \hline
  \end{tabular}
\end{table}


