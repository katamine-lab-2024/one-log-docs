\section{各サイクルで行うこと} \label{sec:process}

サイクルごとに行うことを次に示す. \\

\begin{longtable}[l]{|p{1.5cm}|p{11cm}|}
  \endfirsthead
  \hline
    目的 & 1サイクルにおける機能の開発をガイドする \\
  \hline
    事前条件 &
    \begin{tabular}{l}
      前回のサイクルでの成果物
    \end{tabular} \\
  \hline

  \hline
    計画 &
    \begin{tabular}{l}
      前回のサイクルで作成した成果物を確認する. \\
      規模見積もりをする. \\
      \quad •概念設計 \\
      \quad •WBS \\
      機能をサイクルに分割する. \\
      規模見積もりから, 時間見積もりをする. \\
      スケジュールを作成する. \\
      \quad •ms project \\
    \end{tabular} \\
  \hline
    要求分析 &
    \begin{tabular}{l}
      なんのためにどんなものを作りたいのかを定義する \\
      \quad •システム化の背景 \\
      \quad •システムの目的 \\
      \quad •利用者と、利用者が得られる便益 \\
      利用者がシステムで行うことを定義する. \\
      \quad •システムの全体像 \\
      \quad •ユースケース \\
      実現したいものを整理・分析する. \\
      \quad •要求一覧 \\

      要求に関する画面の要件を作成する \\
      \quad •画面要件 \\
      画面の操作や機能を定義する \\
      \quad •機能要件 \\
      データ要件を定義する \\
      \quad •データ要件 \\
      システムテスト計画を作成する \\
      \quad •機能テスト \\
      \quad •シナリオテスト \\
    \end{tabular} \\
  \hline
    基本設計 &
    \begin{tabular}{l}
      基本設計書のフォーマット作成 \\
      アーキテクチャ設計の作成 \\
      モジュール設計の作成 \\
      \quad •基本設計書 \\
      統合テスト計画の作成
    \end{tabular} \\
  \hline
    詳細設計 &
    \begin{tabular}{l}
      詳細設計書のフォーマット作成 \\
      詳細設計の作成 \\
      \quad •詳細設計書 \\
      単体テスト計画の作成
    \end{tabular} \\
  \hline
    実装 &
    \begin{tabular}{l}
      詳細設計書に基づいて実装 \\
    \end{tabular} \\
  \hline
    テスト &
    \begin{tabular}{l}
      単体テストを行う \\
      結合テストを行う \\
      システムテストを行う \\
    \end{tabular} \\
  \hline
\end{longtable}
