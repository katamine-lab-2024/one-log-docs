\section{Process Script} \label{sec:process}

サイクルごとのプロセスを次に示す. \\

\hspace{-0.9cm}
\begin{tabular}[l]{|p{1.5cm}|p{11cm}|}
  \hline
    目的 & 1サイクルにおける機能の開発をガイドする \\
  \hline
    事前条件 &
    \begin{tabular}{l}
        前回のサイクルでの成果物
    \end{tabular} \\
  \hline

  \hline
    計画 &
    \begin{tabular}{l}
        前回のサイクルでの成果物を分析し, 不足があれば計画に含める. \\
        WBSの作成(初回のみ?) \\
        サイクルの分割(初回のみ?) \\
        サイクル範囲のシステムテスト \\
        概念設計 \\
        機能ごとの規模・時間見積もり \\
    \end{tabular} \\
  \hline
    要求定義 &
    \begin{tabular}{l}
      システム範囲内の要求定義
    \end{tabular} \\
  \hline
    要件定義 &
    \begin{tabular}{l}
      必要な機能?
    \end{tabular} \\
  \hline
    基本設計 &
    \begin{tabular}{l}
      基本設計書のフォーマット作成? \\
      機能を部品に割り当てる \\
      部品間のやり取りを文書化(シーケンス図など) \\
      統合テストの作成、計画?
    \end{tabular} \\
  \hline
    詳細設計 &
    \begin{tabular}{l}
      詳細設計書のフォーマット作成? \\
      詳細設計書の作成 \\
      単体テスト
    \end{tabular} \\
  \hline
    実装 &
    \begin{tabular}{l}
    \end{tabular} \\
  \hline
    テスト &
    \begin{tabular}{l}
      単体からシステムテスト
    \end{tabular} \\
  \hline
\end{tabular}
