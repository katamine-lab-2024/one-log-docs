\section{各サイクルで行うこと} \label{sec:process}

サイクルごとのプロセスを次に示す. \\

\hspace{-0.9cm}
\begin{tabular}[l]{|p{1.5cm}|p{11cm}|}
  \hline
    目的 & 1サイクルにおける機能の開発をガイドする \\
  \hline
    事前条件 &
    \begin{tabular}{l}
      前回のサイクルでの成果物
    \end{tabular} \\
  \hline

  \hline
    計画 &
    \begin{tabular}{l}
      前回のサイクルでの成果物を分析し, 不足があれば計画に含める. \\
      概念設計を行う. \\
      規模見積もりをする. \\
      規模見積もりから, 時間見積もりをする. \\
      サイクルの分割 \\
      サイクル範囲のシステムテスト \\
      機能ごとの規模・時間見積もり \\
    \end{tabular} \\
  \hline
    要求定義 &
    \begin{tabular}{l}
      サイクル範囲内の要求定義
    \end{tabular} \\
  \hline
    要件定義 &
    \begin{tabular}{l}
      必要な機能?
    \end{tabular} \\
  \hline
    基本設計 &
    \begin{tabular}{l}
      基本設計書のフォーマット作成 \\
      アーキテクチャ設計の作成 \\
      モジュール設計の作成 \\
      統合テスト計画の作成
    \end{tabular} \\
  \hline
    詳細設計 &
    \begin{tabular}{l}
      詳細設計書のフォーマット作成 \\

      単体テスト計画の作成
    \end{tabular} \\
  \hline
    実装 &
    \begin{tabular}{l}
      詳細設計書に基づいて実装 \\
    \end{tabular} \\
  \hline
    テスト &
    \begin{tabular}{l}
      単体からシステムテスト
    \end{tabular} \\
  \hline
\end{tabular} \\


成果物については,
