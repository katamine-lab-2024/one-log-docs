\section{各サイクルで行うこと} \label{sec:process}

サイクルごとに行うことを次に示す. \\

\hspace{-0.9cm}
\begin{tabular}[l]{|p{1.5cm}|p{11cm}|}
  \hline
    目的 & 1サイクルにおける機能の開発をガイドする \\
  \hline
    事前条件 &
    \begin{tabular}{l}
      前回のサイクルでの成果物
    \end{tabular} \\
  \hline

  \hline
    計画 &
    \begin{tabular}{l}
      必要があれば, 前回のサイクルで作成した成果物を確認する. \\
      規模見積もりをする. \\
      \quad (概念設計, WBS) \\
      規模見積もりから, 時間見積もりをする. \\
      必要があれば, 機能をサイクルに分割する. \\
      サイクル範囲のシステムテスト \\
      機能ごとの規模・時間見積もり \\
    \end{tabular} \\
  \hline
    要求定義 &
    \begin{tabular}{l}
      なんのためにどんなものを作りたいのかを定義する \\
      \qquad •システム化の背景目的 \\
      \qquad •システムの目的 \\
      \qquad •利用者と、利用者が得られる便益 \\
      システムで利用者が行うことを定義する. \\
      \qquad •システムの全体像 \\
      \qquad •ユーザーストーリー \\
      実現したいものを整理・分析する. \\
      \qquad •要求一覧 \\
      \qquad •データモデル
    \end{tabular} \\
  \hline
    要件定義 &
    \begin{tabular}{l}
      画面に必要な情報と画面遷移を作成する \\
      \qquad •画面遷移図 \\
      画面の操作や機能を定義する \\
      \qquad •機能要件 \\
      画面遷移と機能要件をもとにデータ要件を定義する \\
      \qquad •ER図
    \end{tabular} \\
  \hline
    基本設計 &
    \begin{tabular}{l}
      基本設計書のフォーマット作成 \\
      アーキテクチャ設計の作成 \\
      モジュール設計の作成 \\
      統合テスト計画の作成
    \end{tabular} \\
  \hline
    詳細設計 &
    \begin{tabular}{l}
      詳細設計書のフォーマット作成 \\

      単体テスト計画の作成
    \end{tabular} \\
  \hline
    実装 &
    \begin{tabular}{l}
      詳細設計書に基づいて実装 \\
    \end{tabular} \\
  \hline
    テスト &
    \begin{tabular}{l}
      単体からシステムテスト
    \end{tabular} \\
  \hline
\end{tabular} \\


成果物については,
